\chapter*{}

\section*{Overview of Operation and Computer Technologies Division}
Operation and Computer Technologies Division of RIKEN AICS are responsible for operation and enhancement of the K computer
and the facilities. Operation and support of High Performance Computing Infrastructure (HPCI) system are
also a part of our missions.
Operation and Computer Technologies Division has four teams.
The missions and members of the teams are shown as follows:
\begin{itemize}
%\item Operations and Computer Technologies Division:
%  \begin{itemize}
%  \item Staffs:Division Director, Dupty Division Director (1), Senior Visiting Scientists (2), Student Trainee (1)
\item Facility Operations and Development Team (Team Head: Toshiyuki Tsukamoto)\\
  Missions: Operation and Enhancement of the facility for the K computer\\
  Members: Technical Staff(6)
\item System Operations and Development Team (Team Head: Atsuya Uno)\\
  Missions: Operation and Enhancement of the K computer\\
  Members: R\&D Scientist(5), Technical Staff(2)
\item Software Development Team (Team Head: Kazuo Minami)\\
  Missions: Operation and Enhancement of the K computer software and application tunning\\
  Members: R\&D Scientist(6)
\item HPCI System Development Team (Team Head: Manabu Hirakawa)\\
  Missions: Operation and Enhancement of HPCI system and logistics of the HPCI activity\\
  Members: R\&D Scientist(1), Senior Visiting Scientist(2), Visiting Scientist(1)
\end{itemize}

\section*{Overview of activities in JFY2015}
Three and a half years have passed since the start of the K computer official operation.
(Actually five years have also passed since the start of early access which was a service for limited user.)
Although we suffered from many troubles at the beginning of the operation, many of those have already been fixed
and the recent system operation is relatively stable.
However we experienced a rapid increase of irregular down time caused by the file system failures in JFY2014.
To fix them,  we investigated the causes and developed some workarounds to reduce the impact of
such a system failures.
Then in JFY2015 we achieved shorter irregular down time and higher system availability than that of JFY2014.
On the other hand, at the end of the term, a terrible job congestion occured and waiting time of jobs rapidly increased.
We have already investigated the causes of the event and taken some measures for usage in JFY2016.

In terms of facility operation, we achieved to improve Power Usage Effectiveness (PUE)
which is a major metric to evaluate an energy efficiency of IT center etc.
For the improvements we tried to improve a power generation efficiency and reduction of power consumption for cooling.
Actually, we found that by a comparison between high and middle output level cases, the power generation efficiency of
the gas turbine power generator could be improved more than 30\%.
We also achieved 40\% power saving for air handlers compared with that of in JFY2012 by optimizing some operational
parameters such as number of active handlers, blowout temperatue, number of fans, etc.

In JFY2015, we could have many opportunities to present our research activities.
A major achievements are listed as follows:
\begin{enumerate}
\item Kiyoshi Kumahata and Kazuo Minami. HPCG Performance Improvement on the K computer. HPCG BoF at The International Conference for High Performance Computing Networking Storage and Analysis 2015 (SC15) Austin. 2015.:
This presentation showed a tunning techniques to enhance the performance of HPCG benchmark on the K computer.
\item Kazuo Minami. Power Consumption Reduction Effort for the K computer. Exhibits at The International Conference for High Performance Computing Networking Storage and Analysis 2015 (SC15), Austin. 2015.:
This presentation introdeced a stack of operational improvements for reduction of the power consumption of the K computer.
\item Fumiyoshi Shoji et al. “Long term failure analysis of 10 peta-scale supercomputer”. In: In HPC in Asia session at ISC2015, Frankfurt, Germany, July 12-16. 2015.:
This presentation reported a failure analysis of the K computer and awarded as the Best Poster Award.
\end{enumerate}
